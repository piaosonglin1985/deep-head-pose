%%%%%%%%%%%%%%%%%%%%%%%%%%%%%%%%%%%%%%%%%%%%%%%%%%%%%%%%%%%%%%%%%%%%%%%%%%%%%%%%
%2345678901234567890123456789012345678901234567890123456789012345678901234567890
%        1         2         3         4         5         6         7         8

\documentclass[letterpaper, 10 pt, conference]{ieeeconf}  % Comment this line out if you need a4paper

%\documentclass[a4paper, 10pt, conference]{ieeeconf}      % Use this line for a4 paper

\IEEEoverridecommandlockouts                              % This command is only needed if 
                                                          % you want to use the \thanks command

\overrideIEEEmargins                                      % Needed to meet printer requirements.

%In case you encounter the following error:
%Error 1010 The PDF file may be corrupt (unable to open PDF file) OR
%Error 1000 An error occurred while parsing a contents stream. Unable to analyze the PDF file.
%This is a known problem with pdfLaTeX conversion filter. The file cannot be opened with acrobat reader
%Please use one of the alternatives below to circumvent this error by uncommenting one or the other
%\pdfobjcompresslevel=0
%\pdfminorversion=4

% See the \addtolength command later in the file to balance the column lengths
% on the last page of the document

% The following packages can be found on http:\\www.ctan.org
%\usepackage{graphics} % for pdf, bitmapped graphics files
%\usepackage{epsfig} % for postscript graphics files
%\usepackage{mathptmx} % assumes new font selection scheme installed
%\usepackage{times} % assumes new font selection scheme installed
%\usepackage{amsmath} % assumes amsmath package installed
%\usepackage{amssymb}  % assumes amsmath package installed
\usepackage{graphicx}
\usepackage[caption=false]{subfig}
\usepackage{hyperref}

\title{\LARGE \bf
Report to head pose estimation using CNN
}


\author{Songlin Piao$^{1}$
\thanks{$^{1}$Songlin Piao - 
        {\tt\small piaosonglin1985@hotmail.com}}%
}


\begin{document}



\maketitle
\thispagestyle{empty}
\pagestyle{empty}


%%%%%%%%%%%%%%%%%%%%%%%%%%%%%%%%%%%%%%%%%%%%%%%%%%%%%%%%%%%%%%%%%%%%%%%%%%%%%%%%
\begin{abstract}

This is a technical report about head pose estimation. The baseline paper is from N. Ruiz et al.~\cite{Ruiz2017}. Authors use ResNet50~\cite{He2015} to build a CNN network for estimating head pose (roll, pitch and yaw) from 2D images. This report tried to describe all the possible solutions to improve this baseline method.

\end{abstract}


%%%%%%%%%%%%%%%%%%%%%%%%%%%%%%%%%%%%%%%%%%%%%%%%%%%%%%%%%%%%%%%%%%%%%%%%%%%%%%%%
\section{Topic1}

Authors build a so-called hopenet using ResNet50 as a backbone. They conducted the tests on the AFLW2000 dataset. The results are shown in the Table 1 in the paper ~\cite{Ruiz2017}. Through looking at the source code written by authors, it is found that authors used 66 bins in the range between -99$^\circ$ and 99$^\circ$.
Following solutions could improve the accuracy, but they are not yet proved due to the limited time and training resources.

\begin{itemize}
	\item change the 66 bins to 99 bins and 132 bins
	\item test with ResNet 101 and 152, but this may slow down the speed.
\end{itemize}


\section{Topic2}

In order to make the algorithm satisfy the real time requirement, MobileNet is recommended. I have conducted a test with MobileNet V2. Please have a look at the file train\_mobilenet2.py. The new network is trained with exactly same hyper parameters mentioned in the paper~\cite{Ruiz2017}. It is trained for 25 epochs using Adam optimization with a learning rate of $0.00001$. The training time takes approximately 10.5 hours with GTX 1050Ti Laptop GPU.

\begin{table}[h]
	\caption{Error Comparison between ResNet50 and MobileNet V2 on the AFLW2000 dataset.}
	\label{tb:table_error_compare}
	\begin{center}
		\begin{tabular}{|c|c|c|c|c|}
			\hline
			 & Yaw & Pitch & Roll & MAE \\
			\hline
			Multi-Loss ResNet50 ($\alpha = 1$) & 6.920 & 6.637 & 5.674 & 6.410 \\
			\hline
			Multi-Loss MobileNet-V2 ($\alpha = 1$) & 10.9175 & 7.1786 & 6.6814 &  8.259 \\
			\hline
		\end{tabular}
	\end{center}
\end{table}

Table~\ref{tb:table_error_compare} shows error comparison of ResNet50 and MobileNet V2 on the AFLW2000 dataset. There is slight performance decrease in the case of MobileNet V2. As an opposite, MobileNet V2 has a faster inference speed and very small model size compared to ResNet50 as shown in Table~\ref{tb:table_compare2}.

\begin{table}[h]
	\caption{Speed and Model Size Comparison between ResNet50 and MobileNet V2.}
	\label{tb:table_compare2}
	\begin{center}
		\begin{tabular}{|c|c|c|}
			\hline
			& inference time & model size \\
			\hline
			Multi-Loss ResNet50 ($\alpha = 1$) & 5.5845ms & 95.9Mb \\
			\hline
			Multi-Loss MobileNet-V2 ($\alpha = 1$) & 4.6118ms & 10.1Mb \\
			\hline
		\end{tabular}
	\end{center}
\end{table}

\begin{figure}[ht]
	\centering
	\subfloat[]{\includegraphics[width=0.48\linewidth, height=3.75cm]{mobile1.jpg}} \hspace*{0.5cm} 
	\subfloat[]{\includegraphics[width=0.48\linewidth, height=3.75cm]{hopenet1.jpg}} \\
	\subfloat[]{\includegraphics[width=0.48\linewidth, height=3.75cm]{mobile2.jpg}} \hspace*{0.5cm}
	\subfloat[]{\includegraphics[width=0.48\linewidth, height=3.75cm]{hopenet2.jpg}} \\
	\subfloat[]{\includegraphics[width=0.48\linewidth, height=3.75cm]{mobile3.jpg}} \hspace*{0.5cm}
	\subfloat[]{\includegraphics[width=0.48\linewidth, height=3.75cm]{hopenet3.jpg}} \\
	\caption{The left column are the results from MobileNet V2 and the right column are the results from Hopenet.}
	\label{fig:result_images}
\end{figure}

Fig.~\ref{fig:result_images} shows the visualization of the head pose estimation from MobileNet V2 and Hopenet. Although MobileNet V2 shows higher error in the Table 1, but
it seems in the first two images, the results from MobileNet V2 make more sense than the results from Hopenet. This needs to be further investigated.

\section{Further Improvements}

So far, the input size 224x224 is used for training and testing. It is necessary to test with smaller input sizes such as 168x168 or 112x112.
If the test results are not so bad, this would increase inference speed tremendously.

\section{Newly Added Source Codes}
\begin{itemize}
	\item extract\_valid\_files.py is used for extracting valid samples so that all roll, pitch and yaw are between -99$^\circ$ and 99$^\circ$
	\item mobilenetv2.py is the implementation of MobileNet V2.
	\item train\_mobilenet2.py is the training script using MobileNet V2.
	\item test\_mobilenet2.py is the evaluation script using MobileNet V2.
	\item demo\_mobilenet2\_webcam.py is the demo program to estimate head pose from Webcam using MobileNet V2.
	\item the implementation of MobileNet V2 based model is added inside the file hopenet.py.
	\item MobileNet V1 is also implemented but not used yet.
\end{itemize}

In order to initialize the weights of MobileNet V2, the model mobilenetv2\_1.0-0c6065bc.pth from \url{https://github.com/d-li14/mobilenetv2.pytorch} is used.

{\small
	\bibliographystyle{IEEEtran}
	\bibliography{library}
}

\end{document}
